\documentclass{article}

\usepackage{amsfonts} 
\usepackage{amsmath} 
\usepackage{graphicx} 
\usepackage{float} 
\usepackage{natbib} 
\usepackage{slashbox} 
\usepackage{graphicx} 
\usepackage{flushend} 
\usepackage{amsmath} 
\usepackage{amssymb} 
\usepackage{amsxtra} 
\usepackage{amstext} 
\usepackage{amsthm} 
\usepackage{amsbsy} 
\usepackage{latexsym} 
\usepackage{mathrsfs} 
\usepackage{eucal} 
\usepackage{synttree} 


\usepackage[spanish]{babel}
\usepackage[utf8x]{inputenc}
\author{Flores González Luis Brandon - 312218342 \\ García Argueta Jaime Daniel - 312104739 \\ Tarea 4. Algebra relacional}
\title{}
\date{29 de marzo de 2017}

\begin{document}

	\maketitle	
	
	\begin{enumerate}
		\item Para el problema de la base de datos del Museo que transformaste a Modelo Relacional
		en la tarea anterior, verifica que con éste puedas satisfacer las siguientes consultas:
			\begin{enumerate}
				\item Toda la información de las obras, nombre del artista que la realizó y país de las obras que se
				realizaron con estilo Surrealista o Impresionista.
				
				\item Una lista con el nombre de los artistas y la cantidad de obras que realizó (entre pinturas,
				esculturas y miscelánea).
						
				\item Lista con la cantidad de obras que se tiene por cada estilo (entre pinturas, esculturas y
				miscelánea).
										
				\item Obtener el año en que menos obras se realizaron y la obra más costosa de ese año.
				
				\item Toda la información (obras y artistas) de las obras que se obtuvieron en préstamo el 28 de
				noviembre de año 2014 y que no han sido devueltas.
			\end{enumerate}
		En caso de que el esquema no cubra los puntos anteriores, indica la modificación (o
		modificaciones) que se tendrían que hacer. Incluye Modelo Relacional modificado (sí es el caso).
		Debes indicar la solución en álgebra relacional para las consultas que se solicitan.
		\item Si tienes el siguiente esquema para una Base de Datos:\\

		\textbf{Empleado}(CURP, nombre, calle, ciudad)\\
		\textbf{Trabaja}(CURP, idEmpresa, sueldo)\\
		\textbf{Empresa}(idEmpresa, nombre, ciudad)\\
		\textbf{Jefe}(CURPJ, CURPE)\\
		
		Considera que el sueldo que reciben los empleados es mensual. Escribe una expresión en Álgebra Relacional para cada una de las siguientes consultas:
		
			\begin{enumerate}
				\item Lista con la CURP y nombre de cada empleado que trabaja en Compumundo Hipermega Red(CHR).\\
				
				$\rho_{Empleado(CURP, Empleado.nombre, calle, Empleado.ciudad)}(Empleado)$\\
				$\rho_{Empresa(idEmpresa, Empresa.nombre, Empresa.ciudad)}(Empresa)$\\
				$s \leftarrow \sigma_{Empresa.nombre = 'CHR'}((Empleado \bowtie Trabaja) \bowtie Empresa)$\\
				$\Pi_{CURP, Empleado.nombre}(s)$\\
				
				\item Averiguar el nombre y la ciudad de residencia de todos los empleados que trabajar para el
				Flanders Ship Asociados (FSA).\\
				
				$\rho_{Empleado(CURP, Empleado.nombre, calle, Empleado.ciudad)}(Empleado)$\\
				$\rho_{Empresa(idEmpresa, Empresa.nombre, Empresa.ciudad)}(Empresa)$\\
				$\Pi_{Empleado.nombre, Empleado.ciudad}((Empleado \bowtie Trabaja) \bowtie Empresa)$\\
				
				\item El nombre, la calle y la ciudad de residencia de todos los empleados que trabajan para el CHR
				y ganan más de \$240,000 anuales.\\
				
				$\rho_{Empleado(CURP, Empleado.nombre, calle, Empleado.ciudad)}(Empleado)$\\
				$\rho_{Empresa(idEmpresa, Empresa.nombre, Empresa.ciudad)}(Empresa)$\\
				$s \leftarrow \sigma_{(Empresa.nombre = 'CHR') \land ((sueldo \times 12) > 240 000)}((Empleado \bowtie Trabaja) \bowtie Empresa)$\\
				$\Pi_{Empleado.nombre, Empleado.ciudad, Empleado.calle}(s)$\\
				
				\item Encontrar el nombre y CURP de los empleados que vivan en la misma ciudad en que está
				ubicada la compañía a la que prestan sus servicios.\\
				
				$\rho_{Empleado(CURP, Empleado.nombre, calle, Empleado.ciudad)}(Empleado)$\\
				$\rho_{Empresa(idEmpresa, Empresa.nombre, Empresa.ciudad)}(Empresa)$\\
				$\Pi_{Empleado.nombre, Empleado.CURP}(\sigma_{Empleado.ciudad = Empresa.ciudad}((Empleado \bowtie Trabaja) \bowtie Empresa))$\\
				
				\item Lista con el nombre de los empleados que viven en la misma calle y ciudad que su jefe.\\
				
				$\rho_{Empleado(CURP, Empleado.nombre, calle, Empleado.ciudad)}(Empleado)$\\
				$\rho_{Empresa(idEmpresa, Empresa.nombre, Empresa.ciudad)}(Empresa)$\\
				$J \leftarrow \Pi_{CURPJ}(Jefe)$\\
				$E \leftarrow \Pi_{CURPE}(Jefe)$\\
				$\rho_{J}(CURP)(J)$\\
				$\rho_{E}(CURP)(E)$\\
				$A \leftarrow J \bowtie Empleado$\\			
				$B \leftarrow E \bowtie Empleado$\\
				$A \bowtie B$\\
				
				\item Averiguar la CURP de los empleados que no trabajan para FSA pero si para CHR.\\
				
				$\rho_{Empleado(CURP, Empleado.nombre, calle, Empleado.ciudad)}(Empleado)$\\
				$\rho_{Empresa(idEmpresa, Empresa.nombre, Empresa.ciudad)}(Empresa)$\\
				$f \leftarrow ((Empleado \bowtie Trabaja) \bowtie Empresa)$\\
				$\Pi_{CURP}(\sigma_{(Empresa.nombre = 'CHR') \land \lnot(Empresa.nombre = 'FSA')}(f))$\\
				
				\item Encontrar el nombre, CURP y ciudad de residencia de todos los jefes registrados en la base de
				datos.\\
				
				$\rho_{Empleado(CURP, Empleado.nombre, calle, Empleado.ciudad)}(Empleado)$\\
				$\rho_{Empresa(idEmpresa, Empresa.nombre, Empresa.ciudad)}(Empresa)$\\
				$j \leftarrow \Pi_{CURPJ}(Jefe)$\\
				$\rho_{j(CURP)}(j)$\\
				$\Pi_{CURP, Empleado.nombre, Empleado.ciudad}(j \bowtie Empleado)$\\
				
				\item Una lista con el nombre de todos los empleados que no trabajan para CHR o para FSA.\\
				
				$\rho_{Empleado(CURP, Empleado.nombre, calle, Empleado.ciudad)}(Empleado)$\\
				$\rho_{Empresa(idEmpresa, Empresa.nombre, Empresa.ciudad)}(Empresa)$\\
				$\Pi_{Empleado.nombre}(\sigma_{\lnot(Empresa.nombre = 'CHR' \lor Empresa.nombre = 'FSA')}((Empleado \bowtie Trabaja) \bowtie Empresa))$\\
				
				\item Lista con la CURP de los empleados que ganan más que cualquier empleado CHR.\\
				
				$\rho_{Empleado(CURP, Empleado.nombre, calle, Empleado.ciudad)}(Empleado)$\\
				$\rho_{Empresa(idEmpresa, Empresa.nombre, Empresa.ciudad)}(Empresa)$\\
				$s \leftarrow \sigma_{Empresa.nombre = 'CHR'}((Empleado \bowtie Trabaja) \bowtie Empresa))$\\
				$m \leftarrow \Pi_{max}(CURP \Upsilon max(sueldo)(s))$\\
				$\Pi_{CURP}(\sigma_{sueldo > m}((Empleado \bowtie Trabaja) \bowtie Empresa) - s)$\\
				
				\item Lista con el nombre de las compañías de que están instaladas en cada ciudad en la que haya
				un Krusty Burguer.\\
				
				$\rho_{Empleado(CURP, Empleado.nombre, calle, Empleado.ciudad)}(Empleado)$\\
				$\rho_{Empresa(idEmpresa, Empresa.nombre, Empresa.ciudad)}(Empresa)$\\
				$b \leftarrow \Pi_{idEmpresa}(\sigma_{Empresa.nombre = 'Krusty Burguer'}(Empresa))$\\
				$\Pi_{Empresa.nombre}(b \bowtie Empresa)$\\
				
				\item Borrar toda la información de la compañía Mapple.\\
				
				$\rho_{Empleado(CURP, Empleado.nombre, calle, Empleado.ciudad)}(Empleado)$\\
				$\rho_{Empresa(idEmpresa, Empresa.nombre, Empresa.ciudad)}(Empresa)$\\
				$r \leftarrow \sigma_{Empresa.nombre = 'Maple'}(Empresa)$\\
				$Empresa \leftarrow Empresa - r$\\
				$s \leftarrow \Pi_{idEmpresa}(r \bowtie Trabaja)$\\
				$Trabaja \leftarrow Trabaja - s$\\
				$Empleado \leftarrow Empleado - ((\Pi_{CURP}(s)) \bowtie Empleado)$\\
				
				\item Aumentar el sueldo de los empleados que trabajan en Mr. Plow en un 10\%. \textbf{Revisar :(}\\ 
				
				$a' \leftarrow \Pi_{CURP, ideEmpresa, sueldo}(\sigma_{Empresa.nombre = 'Mr. Plow'}(Trabaja \bowtie Empresa))$
				
				\item Una lista con la cantidad de empleados que se tienen por compañía\\
				
				$\rho_{Empleado(CURP, Empleado.nombre, calle, Empleado.ciudad)}(Empleado)$\\
				$\rho_{Empresa(idEmpresa, Empresa.nombre, Empresa.ciudad)}(Empresa)$\\
				$s \leftarrow Trabaja \bowtie Empresa$\\
				${Empresa.nombre} \hspace{1mm} Y \hspace{1mm} {count(CURP.Trabaja)(s)}$\\
				
				\item Cambiar la ubicación de Ziffcorp (y de todos sus empleados) a Shelbyville.\\
				
				$\rho_{Empleado(CURP, Empleado.nombre, calle, Empleado.ciudad)}(Empleado)$\\
				$\rho_{Empresa(idEmpresa, Empresa.nombre, Empresa.ciudad)}(Empresa)$\\
				$r \leftarrow \sigma_{Empresa.ciudad = 'Ziffcorp'}(Empresa)$\\
				$Empresa \leftarrow Empresa - r$\\
				$s \leftarrow (\Pi_{idEmpresa, Empresa.nombre}(r) \times 'Shelbyville')$\\
				$Empresa \leftarrow Empresa \cup S$\\
				$h \leftarrow \sigma_{Empresa.ciudad = 'Sherbyville}((Empleado \bowtie Trabaja) \bowtie Empresa)$\\
				$n \leftarrow \Pi_{CURP, Empleado.nombre}(h) \times ('null', 'Shellbyville')$\\
				$Empleado \leftarrow Empleado - \Pi_{CURP, Empleado.nombre, calle, Empleado.ciudad}(h)$\\
				$Empleado \leftarrow Empleado \cup n$
				
				\item
				\item A los empleados que trabajan en Sorny y que ganen \$10,000 mensuales hacer un incremento
				del 7\%, mientras que a los que trabajan en Panaphonics y que ganen más de \$15,000
				mensuales reducir su sueldo en un 8\%.\\
				
				$\rho_{Empleado(CURP, Empleado.nombre, calle, Empleado.ciudad)}(Empleado)$\\
				$\rho_{Empresa(idEmpresa, Empresa.nombre, Empresa.ciudad)}(Empresa)$\\
				$s \leftarrow \Pi_{CURP,idEmpresa, sueldo}(\sigma_{Empresa.nombre = 'Sony' \land sueldo = 10000}((Empleado \bowtie trabaja) \bowtie Empresa))$\\
				$p \leftarrow \Pi_{CURP, ideEmpresa, sueldo}(\sigma_{Empresa.nombre = 'Panasonic' \land sueldo > 15000}((Empleado \bowtie trabaja) \bowtie Empresa))$\\
				$Trabaja \leftarrow Trabaja - s$\\
				$Trabaja \leftarrow Trabaja - p$\\
				$Trabaja \leftarrow Trabaja\cup \Pi_{CURP, ideEmpresa, sueldo + sueldo \times .08}(p)$\\
				
				\item Lista de los empleados que trabajan en más de dos compañías y el número de compañías en
				que laboran.\\
				
				$s \leftarrow CURP \hspace{1mm} Y \hspace{1mm} count(idEmpresa)(Trabaja)$\\
				$f \leftarrow \sigma_{count \geq 2}(s)$\\
				$Empleado \bowtie f$\\
				
				\item Lista que muestre la CURP del jefe y el número de empleados que están a su cargo, agrupados
				por compañía.
				
				
				
				\item Una lista de los empleados que ganan más de \$140,000 anuales y que no viven en Springfield.\\
				
				$\rho_{Empleado(CURP, Empleado.nombre, calle, Empleado.ciudad)}(Empleado)$\\
				$\rho_{Empresa(idEmpresa, Empresa.nombre, Empresa.ciudad)}(Empresa)$\\
				$\Pi_{CURP}(\sigma_{(sueldo \times 12) > 140 000 \land \lnot(Empleado.ciudad = 'Springfield')})$\\
				
				\item La empresa que paga el mayor sueldo promedio.\\
				
				$p \leftarrow (idEmpresa \hspace{1mm} Y \hspace{1mm} avg(sueldo)(Trabaja))$\\
				$\Pi_{idEmpresa}((idEmpresa \hspace{1mm} Y \hspace{1mm} max(avg)(p)))$\\
				
				\item Moe Szyslak decide dejar su bar y entrar a trabajar a la planta nuclear, siendo su nuevo jefe
				Carl Carlson. Refleja estos cambios en la base datos.\\
				
				Como Moe Szyslak no esta en la base de datos, necesitamos insertar.\\
				
				$\rho_{Empleado(CURP, Empleado.nombre, calle, Empleado.ciudad)}(Empleado)$\\
				$\rho_{Empresa(idEmpresa, Empresa.nombre, Empresa.ciudad)}(Empresa)$\\
				$Empleado \leftarrow Empleado \cup ('8541521456', 'Moe Szyslak', 'Una calle', 'Springfield')$\\
				
				Además suponemos que solo hay un Carl Carlson.\\
				
				$n \leftarrow \Pi_{CURP}(\sigma_{Empleado.nombre = 'Carl Carlson'}(Empleado))$\\
				$\rho_{n(CURPE)}(n)$\\
				$Jefe \leftarrow \Pi_{CURPJ}(Jefe) \times n$
							
			\end{enumerate}
	\end{enumerate}	
\end{document}